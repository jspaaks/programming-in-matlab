\chapter{Introduction}
\thispagestyle{fancy}

When you watch the weather forecast, all predictions are based on the results of computer models. With real-time input of meteorological conditions and a model that simulates (part of) the global climate system, it is possible to predict the weather for the near future. Of course these models can never be 100 percent accurate, as everybody who has gone out in shorts and ended up soaked with rain can confirm.

In another example of modeling, the United Nations use computer models to predict the development of global phenomena like diseases, famine and death. Other model simulations based on demographics and population data can be used to estimate future trends in world population and migration.

Science has made tremendous progress in the last decades. Measuring techniques, instruments and knowledge transfer are becoming better and more efficient. But the examples above show the importance of another factor: the increasing use of computers to analyze and generate data. It is now possible to quickly access, analyze and manipulate amounts of data on your computer that would have taken up complete libraries in the past. And the possibilities reach beyond simply using existing data: models can approximate real-life processes and generate data that was not available before. Furthermore, models can deepen the understanding of the earth and all of the processes influencing our environment.

The computer programs you may have so far encountered in your life - Microsoft Excel, Word and PowerPoint, ArcGIS and Stella - are all packages for specific tasks. Excel, ArcGIS and Stella allow you to analyze, manipulate and visualize data to some degree. However, you are always restricted to the features that the software developers included in these packages. 

Alternatively, programming languages, such as Fortran, Basic, C, C++, Pascal, Turbo Pascal, Object Pascal (Delphi), R, Python and \MATLAB{} give you the freedom to write your own programs that simulate and calculate almost anything you want them to. Many specific scientific problems cannot be tackled without a custom-made computer program. 

In this course you will use MATLAB, a `next-generation' programming language based on C and Fortran, but without all the programming overhead that comes with those languages, making MATLAB much easier to use in comparison.

At the University of Amsterdam the generation and transfer of programming knowledge has become an important feature of the educational program. It is the believe---and rightly so---that a student who is able to write his own computer programs, and understand those of others, has more flexibility when analyzing, manipulating and understanding data. With Excel you cannot realistically model global climate change or any other (complicated) earth process. At this moment, MATLAB is being used to model hydrological systems, erosion in the semi-arid Mediterranean, rockfall in the Alps, the interaction between soil and water, or bird migration from Scandinavia to Africa, to name but a few current research topics where programming is involved.

\MATLAB{}'s user-friendly interface and programming flexibility have made it a popular tool for scientists. On the Internet, large amounts of \MATLAB{} programs and tutorials can be found. Furthermore, MATLAB is an `array-oriented' programming language and therefore very suitable for spatial (1D, 2D, 3D, 4D\dots) calculations.

Because of all this, we think it is important that a student in Earth Sciences at the University of Amsterdam be introduced to \MATLAB{} at an early stage. This course, Programming in \MATLAB{}, has been developed to make you comfortable with programming in \MATLAB{} and teach you how the Earth's processes can be converted into \MATLAB{} programs.

\vspace{2em}

\noindent \textbf{Take your time going through the exercises. Make sure you try to interpret your results. If you rush too much, you won't be able to pick up the information and you will easily forget what you have learned. }


